\begin{abstract}
    Tiểu luận này trình bày một số phương pháp lặp giải hệ phương trình tuyến tính
    lớn bằng cách sử dụng kĩ thuật tối thiểu hóa phần dư tại mỗi bước lặp (minimal residual
    (MR)). Ma trận của hệ tuyến tính được xem xét ở dạng tổng quát (sử dụng 
    phương pháp GMRES (Generalized minimal residual method)) và dạng đối xứng 
    xác định dương (sử dụng phương pháp MINRES (Minimal residual method)). 
    Tiệu luận tập trung vào giải quyết bài toán ma trận đối xứng như một trường hợp riêng (lấy
    ý tưởng từ việc giải bài toán tổng quát). Kỹ thuật MINRES được sử dụng như 1 cách hiệu quả
    để giảm thiểu số bước lặp cũng như không gian lưu trữ khi so sánh với GMRES.
    Không gian con Krylov được xem xét để tìm
    nghiệm gần đúng của hệ. Bài này cũng giới thiệu về phương pháp lặp Arnoldi và Lanczos.
    Cuối cùng ta sử dụng ngôn ngữ lập trình Python để minh họa các thuật giải và so sánh hiệu năng 
    của chúng. 
\end{abstract}