\section{Giải thuật Lanczos}
Khi ma trận $A$ là một ma trận Hermintian (trong không gian thực ta có thể 
nói $A$ là ma trận đối xứng), giải thuật Arnoldi được trình bày ở phần trước 
có thể được đơn giản hóa bằng giải thuật Lanczos. Các phương trình cho hệ số 
hơi khác một chút (nhưng vẫn tương đương về mặt toán học) thường được dùng 
cho trường hợp ma trận Hermintian.

\textbf{Lanczos Algorithm} (for Hermintian matrices A)
\begin{lstlisting}[style=algo]
    Given $q_1$ with $\|q_1\| = 1$, set $\beta_0 = 0$
    For $j = 1, 2, ...$
        $\tilde{q}_{j + 1} = Aq_j - \beta_{j - 1}q_{j - 1}$
        set $\alpha_j = \langle\tilde{q}_{j+1}, q_j\rangle$
        $\tilde{q}_{j+1} \gets \tilde{q}_{j+1} - \alpha_{j}q_j$
        $\beta_{j} = \|\tilde{q}_{j+1}\|$
        $q_{j + 1} = \tilde{q}_{j+1}/\beta_{j}$
\end{lstlisting}

Một cách trực quan ta có thể thấy ngay rằng số vòng lặp đã được giảm đi một bậc so 
với giải thuật Arnoldi. Đây thực chất không phải điều "kì diệu" mà do ta đang xem xét
bài toán trong trường hợp ma trận $A$ là đối xứng dẫn đến việc ma trận Hessenberg $H$ 
là "thưa" hơn rất nhiều so với trường hợp tổng quát.

Có thể thấy rằng các vectors được hình thành 
trong giải thuật trên là tương đương với các vectors trong giải thuật Arnoldi 
khi ma trận $A$ là ma trận Hermintian. Ngoài ra chúng ta cũng cần chỉ ra rằng 
các vectors $q_j$ cũng lập thành một cơ sở trực chuẩn cho không gian Krylov (sinh 
bởi $A$ và $q_1$).

Ta cùng chứng minh khẳng định này bằng phương pháp quy nạp như sau.

Trước hết ta thấy rằng từ cách chọn vectors $q$ và hệ số $\beta$, các vectors 
này nằm trong không gian Krylov và mỗi vector đều có chuẩn bằng 1.

Ta có:
\begin{equation*}
    \begin{split}
        \alpha_j &= \langle \tilde{q}_{j+1}, q_j \rangle \\
        \leftrightarrow 0 &= \langle \tilde{q}_{j+1}, q_j \rangle - \alpha_j \langle q_j, q_j \rangle \\
        &= \langle \tilde{q}_{j+1} - \alpha_j q_j, q_j \rangle = \beta_j \langle q_{j+1}, q_j \rangle
    \end{split}
\end{equation*}

Hay $\langle q_{j+1}, q_j \rangle = 0$. Giả sử rằng $\langle q_k, q_i \rangle = 0$ với mọi 
$i \neq k$ mà $k,i \leq j$. Ta có:

\begin{equation*}
    \begin{split}
        \langle \tilde{q}_{j+1}, q_{j-1} \rangle &= \langle Aq_j - \alpha_jq_j - \beta_{j-1}q_{j-1}, q_{j-1} \rangle 
        = \langle Aq_j, q_{j-1} \rangle - \beta_{j-1} \\
        &= \langle q_j, Aq_{j-1} - \beta_{j-1} \rangle \\
        &= \langle q_j, \tilde{q}_j + \alpha_{j-1}q_{j-1} + \beta_{j-2}q_{j-2} \rangle - \beta_{j-1}
        = \langle q_j, \tilde{q}_j \rangle - \beta_{j - 1} = 0
    \end{split}
\end{equation*}

Với $i < j - 1$, Ta có (lưu ý rằng $A = A^T$):
\begin{equation*}
    \begin{split}
        \langle \tilde{q}_{j+1}, q_{i} \rangle &= \langle Aq_j - \alpha_jq_j - \beta_{j-1}q_{j-1}, q_{i} \rangle 
        = \langle Aq_j, q_i \rangle \\
        &= \langle q_j, Aq_i \rangle \\
        &= \langle q_j, \tilde{q}_{i+1} + \alpha_iq_i + \beta_{i - 1}q_{i - 1} \rangle = 0
    \end{split}
\end{equation*}
Như vậy tập các vectors $(q_1, ..., q_{j+1})$ lập thành 1 cơ sở trực chuẩn của không gian Krylov 
$span(q_1, Aq_1, ..., A^jq_1)$.

Giải thuật Lanczos có thể viết gọn dưới dạng ma trận như sau:
\begin{equation}
    AQ_k = Q_kT_k + \beta_kq_{k+1}e_k^T = Q_{k+1}T_{k+1, k}
\end{equation}
Trong đó $Q_k$ là ma trận $m \times k$ gồm $k$ cột đầu tiên của các vectors cơ sở trực chuẩn 
$q_1, ..., q_k$, $e_k$ là vectors đơn vị thứ $k$ và $T_k$ là ma trận Hermintian 
\textit{3 đường chéo} (tridiagonal) với các hệ số như sau:

\begin{equation}
    matrix here
\end{equation}
Ma trận $(k+1) \times k$ $T_{k+1,k}$ là ma trận với $T_k$ là khối $k \times k$ phía trên và
cột cuối cùng là $\beta_k e_k^T$.