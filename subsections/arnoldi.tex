\section{Thuật giải Arnoldi}
Phương pháp GMRES sử dụng quá trình Gram-Schmitz chỉnh sửa để thiết lập
một cơ sở trực chuẩn cho không gian Krylov $span\{r_0, Ar_0, ..., A^kr_0 \}$.
Quá trình Gram-Schmitz được áp dụng cho không gian này được gọi là phương pháp \textit{Arnoldi}
Giải thuật này nhằm phân tích $A=QHQ^*$ hoặc $AQ = QH$, trong đó $Q$ là ma trận trực chuẩn (unitary) 
còn $H$ là ma trận Hessenberg\footnote{
    \textit{
    Ma trận $H$ vuông kích thước $n \times n$ được gọi là ma trận upper Hessenberg nếu
    $h_{ij} = 0$ với mọi $i,j$ mà $i > j + 1$
}
}.

Gọi $Q_n$ (ma trận $m \times n$) là ma trận được tạo thành bởi $n$ cột đầu tiên của ma trận $Q$

\begin{equation}
    Q_n = \begin{bmatrix}
        q_1 & | & q_2 & | & q_3 & ... & | & q_n
    \end{bmatrix}    
\end{equation}

Gọi $\tilde{H}_n$ là phần trên bên trái $(n + 1) \times n$ của ma trận $H$.
$\tilde{H}_n$ đương nhiên cũng là 1 ma trận Hessenberg. 

Ta có phương trình:

\begin{equation}
    AQ_n = Q_{n + 1}\tilde{H}_n \label{arnoldi:1}
\end{equation}

Ta có thể viết riêng cho cột thứ $n$ của phương trình như sau:

\begin{equation}
    Aq_n = h_{1n}q_1 + ... + h_{nn}q_n + h_{n+1,n}q_{n + 1}
\end{equation}

Có nghĩa là $q_{n+1}$ có thể tìm được từ $n$ vectors $q_1, q_2, ..., q_n$. 
$q_{n+1}$ được tính theo phương pháp lặp bằng giải thuật dưới đây (thuật giả Arnoldi)


\textbf{Arnoldi Algorithm}
\begin{lstlisting}[style=algo]
    Given $q_1$ with $\|q_1\| = 1$
    For $j = 1, 2, ...$
        $\tilde{q}_{j + 1} = Aq_j$
        For $i = 1, ..., j$
            $h_{ij} = \langle\tilde{q}_{j+1}, q_j\rangle$
            $\tilde{q}_{j+1} \gets \tilde{q}_{j+1} - h_{ij}q_i$
        $h_{j+1,j} = \|\tilde{q}_{j+1}\|$
        $q_{j + 1} = \tilde{q}_{j+1}/h_{j+1,j}$
\end{lstlisting}

Trong đó $\langle v, u \rangle$ là tích vô hướng của 2 vectors $v$ và $u$.

Thực chất, ta có thể chứng minh phương trình \eqref{arnoldi:1} bằng cách sử dụng quy nạp và tính chất
của ma trận Hessenberg dựa vào cách chọn các vectors $q_j$ và $h_{ij}$ trong thuật giải Arnoldi ở trên 
(chi tiết trong phần phụ lục).


Chúng ta có thể tìm được ma trận $Q$ với các ngôn ngữ lập trình bậc cao như MATLAB một cách 
tương đối dễ đàng. Ma trận $A$ ở đây chỉ xuất hiện trong tích $Aq_j$ và MATLAB cũng tích hợp 
sẵn chương trình nhân ma trận. Tuy nhiên trong bài này, tác giả sử dụng ngôn ngữ lập trình pyhthon 
cũng là một ngôn ngữ bậc cao hiện đại và có nhiều bộ thư viện giúp ta tính toán với ma trận tương 
đối dễ dàng và tường minh như \textit{numpy} chẳng hạn.
