\section{GMRES}
Trong phương pháp GMRES, nghiệm gần đúng $x_k$ được chọn dưới dạng 
$x_k = x_0 + Q_k y_k$ với một vector $y_k$ nào đó cần tìm. Nghĩa là $x_k$ là $x_0$
cộng với một tổ hợp tuyến tính của các vectors cơ sở trực chuẩn cho không gian Krylov.
Chúng ta sẽ tìm vector $y_k$ sao cho phần dư $r_k = r_0 - AQ_k y_k$ có chuẩn-2 tối thiểu.
Như vậy vector $y$ phải thỏa mãn bài toán bình phương tối thiểu:
\begin{equation*}
    \begin{split}
        &\min_{y} \|r_0 - AQ_k y \| = \min_y \|r_0 - Q_{k+1}H_k y\| \\
        &= \min_{y} \|Q_{k+1}(\beta e_1 - H_k y)\| = \min_{y} \|\beta e_1 - H_k y\|
    \end{split}
\end{equation*}
Trong đó, $\beta = \|r_0\|$, $e_1$ là vector đơn vị $(k+1)$ chiều $(1,0,...,0)^T$.

Từ đây, ta có thể đưa ra những bước cơ bản để thực hiện giải nghiệm gần đúng $x_k$ 
bằng phương pháp GMRES như sau:

1. Lấy một điểm bắt đầu, vector $x_0$, tính phần dư $r_0 = b - A_x0$, đặt 
$q_1 = r_0 / \|r_0\|$

2. Cho $k$ chạy từ 1: $k=1,2,...$: tính $q_{k+1}$ và $h_{i,k}$ với 
$i = 1,2,...,k+1$ sử dụng giải thuật Arnoldi.

3. Tính $x_k = x_0 + Q_k y_k$ với $y_k$ là nghiệm của bài toán bình phương tối thiểu 
$\min_y \|\beta e_1 - H_k y\|$

Việc giải bài toán GMRES tổng quát được mô tả ở trên là khá tốn công sức và thiếu thực tế
vì khối lượng tính toán cũng như không gian lưu trữ khá lớn, đặc biệt khi kích thước của hệ 
là lớn. Ta nhận thấy số vòng lặp $\sim k^2$ tuy nhiên tại thời điểm này ta chưa nhận định được 
sau bao nhiêu bước $k$ thì thu được nghiệm gần đúng chấp nhận được. Ta có thể dùng giải thuật
GMRES(j) được định nghĩa đơn giản là \textit{khởi động lại} GMRES qua mỗi $j$ bước, sử dụng 
bước lặp gần nhất làm giá trị khởi tạo cho vòng tiếp theo. Chúng ta sẽ không bàn quá sâu về  GMRES
ở đây nữa vì trọng tâm phần này, tác giả muốn giới thiệu về ý tưởng chính của phương pháp và tìm 
ra 1 cách hiệu quả để giải bài toán ma trận đối xứng. Chi tiết cho bài toán bình phương tối thiểu 
(\textit{least square}) cũng sẽ được trình bày ở phần sau.