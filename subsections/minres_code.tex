\section{MINES code}
Để có cái nhìn trực quan về giải thuật MINRES, chúng ta sẽ dùng python với
thư viện scipy để giải thử một hệ phương trình với ma trận đối xứng kích thước $20 \times 20$.

Thư viện scipy đã cung cấp hàm dựng sẵn để giải hệ, ta cùng xem đầu ra và xem xét tính hội tụ của phần dư:

\begin{lstlisting}[style=code]
    import numpy as np
    from scipy.sparse.linalg import minres

    np.random.seed(100)
    A = np.random.randint(1, 10, (20,20))
    A = A + A.T
    print(A)
    b = np.random.randint(1, 10, 20)
    x, exitCode = minres(A, b, maxiter=1000, show=True)
    print(exitCode)
\end{lstlisting}
Output:
\lstinputlisting[numbers=none, basicstyle=\tiny]{assets/output.txt}
\section*{Phân tích kết quả}
Như ta thấy trong code, tham số maxiter được set là $1000$ nghĩa là ta cho phép tối đa $1000$ vòng lặp.
Tolerent mặc định của hàm minres trong thư viện scipy được lấy là $10^{-5}$. rnorm là phần dư có giá trị là $1.7053e-06$ 
và đạt được sau 21 vòng lặp. Nếu so sánh với phương pháp khử Gauss, số bước tính $\sim 20^3 = 8000$ 
