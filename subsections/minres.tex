\section{MINRES}
Như đã trình bày ở phần trên, ý tưởng chung cho GMRES hay MINRES là tìm cách xấp
xỉ nghiệm $x_k$ biểu diễn trong không gian Krylov với chuẩn-2 của phần dư là cực tiểu.

Nghĩa là nếu $q_1$ được chọn trong thuật giải Lanczos là $r_0/\beta$ với 
$\beta = \|r_0\|$ thì nghiệm gần đúng sẽ có dạng $x_k = x_0 + Q_ky_k$. Trong đó
$y_k$ được tính toán để làm cực tiểu hóa phần dư. Nói cách khác, $y_k$ là nghiệm
của bài toán bình phương tối thiểu:
\begin{equation}
    \begin{split}
        \min_y \|r_0 - AQ_ky\| &= \min_y \|r_0 - Q_{k+1}T_{k+1, k}y\| \\
        &= \min_y \|Q_{k+1}(\beta e_1 - T_{k+1,k}y)\| \\
        &= \min_y \|\beta e_1 - T_{k+1,k}y\|
    \end{split}
\end{equation}
tương tự như thuật giải GMRES ở phần trước. Tuy nhiên phương trình này đơn giản hơn nhiều 
vì ma trận $T_{k+1,k}$ là rất thưa khi so sánh với ma trận $H_k$.

Giải thuật MINRES cũng không yêu cầu cần phải lưu trữ các vectors cơ sở trực chuẩn 
khi dùng Lanczos (so sánh với Arnoldi). 

Tiếp theo, ta sẽ cùng tìm cách giải bài toán bình phương tối thiểu này để tìm nghiệm 
gần đúng $x_k$.

Gọi $R_{k \times k}$ là khối $(k \times k)$ phía trên của ma trận $R$ trong phân tích $QR$
của ma trận $T$ ($T_{k+1, k} = F^HR$). Trong đó ma trận $F^H$ là ma trận trực chuẩn (unitary)
kích thước $(k+1) \times (k+1)$ và ma trận $R$ là ma trận tam giác trên kích thước $(k+1) \times k$
với khối $k \times k$ và các hàng dưới đều bằng $0$.

Đối với trường hợp đặc biệt này (ma trận $T_{k+1,k}$ - tridiagonal) thì $R_{k \times k}$ chỉ có 2 đường chéo khác $0$.
Ta định nghĩa $P_k \equiv (p0, ..., p_{k-1}) \equiv Q_kR_{k \times k}^{-1}$. 
Do đó, $p_0$ là tích của $q_1$ với các cột của ma trận $P_k$ có thể được tính như sau:
\begin{equation}
    p_{k-1} = \left(q_k - b_{k-2}^{(k-1)} p_{k-2} - b_{k-3}^{(k-1)} p_{k-3}\right) / b_{k-1}^{(k-1)}
\end{equation}
Trong đó $b_{k-1}^{(k-l)}$ là phần tử $(k-l+1,k)$ của ma trận $R_{k \times k}$. Lưu ý rằng các phần tử
$b_{k-1}^{(k-l)}$ là khác $0$. Nghiệm gần đúng $x_k$ được cập nhật từ $x_{k-1}$ như sau:
\begin{equation}
    x_k = x_0 + P_k \beta (Fe_1)_{k \times 1} = x_{k-1} + a_{k-1}p_{k-1}
\end{equation}
Với $a_{k-1}$ là phần tử thứ $k$ của $\beta (Fe_1)$.
Ta có giải thuật MINRES được viết chi tiết như sau:

\textbf{MINRES Algorithm} (for Hermintian matrices A)
\begin{lstlisting}[style=algo]
    Given vector $x_0$ 
    Compute $r_0 = b - Ax_0$
    Set $q_1 = r_0 \|r_0\|$
    Initialize $e = (1,0,...,0)^T$, $\beta = \|r_0\|$
    For k = 1,2,...
        Compute $q_{k+1}$, $\alpha_k \equiv T(k,k)$, $\beta_k \equiv T(k+1, k) \equiv T(k, k+1)$
        using Lanczos algorithm

        Apply $F_{k-2}$ and $F_{k-1}$ to the last column of $T$:

            $\begin{pmatrix}
                T(k-2, k) \\
                T(k- 1, k)
            \end{pmatrix}
            \gets 
            \begin{pmatrix}
                c_{k-2} & s_{k-2} \\
                -\bar{s}_{k-2} & c_{k-2}
            \end{pmatrix}
            \begin{pmatrix}
                0 \\
                T(k- 1, k)
            \end{pmatrix}$, if $k>2$

            $\begin{pmatrix}
                T(k-1, k) \\
                T(k, k)
            \end{pmatrix}
            \gets 
            \begin{pmatrix}
                c_{k-1} & s_{k-1} \\
                -\bar{s}_{k-1} & c_{k-1}
            \end{pmatrix}
            \begin{pmatrix}
                T(k- 1, k) \\
                T(k, k)
            \end{pmatrix}$, if $k>1$

        Compute the kth rotation, $c_k$ and $s_k$ to annihilate 
        the $(k+1,k)$ entry of $T^2$

        Apply kth rotation to $e$ and to the last column of $T$
            $\begin{pmatrix}
                e(k) \\
                e(k+1)
            \end{pmatrix}
            \gets 
            \begin{pmatrix}
                c_{k} & s_{k} \\
                -\bar{s}_{k} & c_{k}
            \end{pmatrix}
            \begin{pmatrix}
                e(k) \\
                0
            \end{pmatrix}$

            $T(k,k) \gets c_kT(k,k) + s_kT(k+1, k)$
            $T(k+1, k) \gets 0$

        Commpute $p_{k-1} = \left[ q_k - T(k-1,k)p_{k-2} - T(k-2,k)p_{k-3} \right]/T(k,k)$
        where undefined tearms are zero for $k \leq 2$

        Set $x_k = x+{k-1} + a_{k-1}p_{k-1}$, where $a_{k-1} = \beta e(k)$
\end{lstlisting}